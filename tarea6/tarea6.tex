\documentclass[12pt, a4paper]{article}

\usepackage[utf8]{inputenc}
\usepackage[spanish]{babel}
\usepackage{amsmath}
\usepackage{amsfonts}
\usepackage{amssymb}
\usepackage{graphicx}
\usepackage{float}
\usepackage{listings}
\usepackage{multirow}
\usepackage{color}

\usepackage[left=2cm,right=2cm,top=2cm,bottom=2cm]{geometry}

\author{5175: \'Angel Moreno \\ Tarea \# 6: Complejidad asintótica experimental}
\title{Optimizaci\'on flujo en redes}

\begin{document}
\maketitle

\section*{Seleccion de proyecto}

Sea $P$ el conjunto de los posibles proyectos. Cada proyecto $v \in P$ tiene asociado un ingreso $p_{v}$. Algunas combinaciones de proyectos generan beneficio y otros generan costo. Algunos proyectos no pueden empezar hasta que que otros proyectos esten terminados. Sea $D$ es el conjunto de dependencia que es un grafo aciclico, donde la arista $i \to j$ indica que el proyecto $i$ depende de $j$. El problema consiste en encontrar un subconjunto de proyectos donde el ingreso sea maximo, en dado caso si el ingreso es negativo la solucion es no hacer nada.

Definimos un nuevo gráfico G agregando un vértice de origen $s$ y un vértice de destino $t$ del conjunto $D$, con la arista $s \to t$ para cada trabajo rentable (con $p_j> 0$), y con la arista $i \to t$ un costo de
trabajo (con $p_i <0$). Asignamos capacidades de borde de la siguiente manera:
\begin{itemize}
\item $c(s \to j) = p_{j}$ para todo trabajo $j$ rentable
\item $c(i \to t) = -p_{i}$ para todo costo de trabajo $i$
\item $c(i \to j) = \infty$ para todo arista dependiente $i \to j$
\end{itemize}

Para todo subconjunto $A$ de proyectos, se definen tres funciones:
\[ cost(A) := \sum_{i \in A: p_i < 0} -p_i = \sum_{i \ in A} c(i \to j) \]
\[ benefit(A) := \sum_{j \in A: p_i > 0} p_j = \sum_{j \in A} c(s \to j) \]
\[ profit(A) := \sum_{i \in A} p_i = benefit(A) - cost(A) \]


\end{document}